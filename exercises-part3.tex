\subsection{Exercise 4.1: Quicksort Partitioning Worst Case}
\textbf{Problem:} Prove that the worst case in Partitioning Algorithm (for Quicksort) has running time $\Theta(n^2)$, where $n$ is the cardinality of the set of elements in the partitioning.

\textbf{Solution:} We provide both an intuitive proof and a formal proof by induction.

\textbf{Part 1: Intuitive Proof}
\begin{enumerate}[leftmargin=*,noitemsep]
    \item \textbf{Worst Case Scenario:}
    \begin{itemize}[noitemsep]
        \item At each step, we get maximally unbalanced partitions:
        \item A $k-1$ element array and an empty array
        \item This happens when pivot is always smallest/largest element
    \end{itemize}

    \item \textbf{Recurrence Relation:}
    Let $T(n)$ be the running time of Quicksort with Partition:
    \begin{itemize}[noitemsep]
        \item Splitting time is linear: $\Theta(k)$ for array of size $k$
        \item Base case: $T(0)$ is constant, so $T(0) = \Theta(1)$
        \item For size $k$: $T(k) = T(k-1) + T(0) + \Theta(k)$
        \item Simplifies to: $T(k) = T(k-1) + \Theta(k)$
    \end{itemize}

    \item \textbf{Solving the Recurrence:}
    \begin{align*}
        T(n) &= T(n-1) + \Theta(n) \\
        &= T(n-2) + \Theta(n-1) + \Theta(n) \\
        &= T(n-3) + \Theta(n-2) + \Theta(n-1) + \Theta(n) \\
        &\vdots \\
        &= T(0) + \Theta(1) + \Theta(2) + ... + \Theta(n-1) + \Theta(n)
    \end{align*}

    \item \textbf{Final Step:}
    \begin{itemize}[noitemsep]
        \item Sum is arithmetic series: $1 + 2 + ... + (n-1) + n$
        \item Using identity: $1 + 2 + ... + n = \frac{n(n+1)}{2}$
        \item Therefore: $T(n) = \Theta(\frac{n(n+1)}{2}) = \Theta(n^2)$
    \end{itemize}
\end{enumerate}

\textbf{Part 2: Formal Proof by Induction}
\begin{enumerate}[leftmargin=*,noitemsep]
    \item \textbf{Claim:} $T(n) = \Theta(n^2)$ for worst-case running time
    
    \item \textbf{Precise Statement:}
    \begin{itemize}[noitemsep]
        \item For all $0 < m < n$: $T(m) = \Theta(m^2)$
        \item This means $\exists c_1,d_1 > 0: c_1m^2 \leq T(m) \leq d_1m^2$
        \item Partition time $P(m) = \Theta(m)$, so $\exists c_2,d_2 > 0: c_2m \leq P(m) + T(0) \leq d_2m$
    \end{itemize}

    \item \textbf{Constants:}
    \begin{itemize}[noitemsep]
        \item Let $c = \min\{c_1,c_2\}$ and $d = \max\{d_1,d_2,1\}$
        \item Then for all $m \geq n-1$: $cm^2 \leq T(m) \leq dm^2$
        \item And for all $m \geq n$: $2cm \leq T(0) + P(m) \leq dm$
    \end{itemize}

    \item \textbf{Inductive Step:}
    \begin{align*}
        T(n) &= T(n-1) + T(0) + P(n) \\
        c(n-1)^2 + 2cn &= cn^2 - 2cn + 1 + 2cn = cn^2 + 1 > cn^2 \\
        d(n-1)^2 + dn &= dn^2 - 2dn + 1 + dn = dn^2 - dn + 1 \leq dn^2
    \end{align*}

    Therefore: $cn^2 < T(n) \leq dn^2$, proving $T(n) = \Theta(n^2)$
\end{enumerate}

\textbf{Key Insights:}
\begin{itemize}[noitemsep]
    \item The worst case occurs with extremely unbalanced partitions
    \item Each partition step costs linear time
    \item The cumulative effect leads to quadratic runtime
    \item Both intuitive and formal proofs confirm $\Theta(n^2)$ complexity
\end{itemize}
