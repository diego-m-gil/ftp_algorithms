\newpage

% Mock Exam Part 1

\subsection{Exercise 1: Running Time}

\subsubsection*{Example 1}
1. (6P) Write the following Theta-classes in non-decreasing order from left to right (i.e., smallest on the left):

\(\Theta(n^2), \Theta(e^{\log_2 n}), \Theta(\sin n), \Theta(\ln^2 n), \Theta(\log_2 n), \Theta(n^3 + \ln(n^2)), \Theta(e^{n^2}), \Theta(e^{\ln(n^4)})\)

Solution: Understand growth rates: \(\sin n\) is slowest, followed by logarithmic, polynomial, and exponential. Use limits to compare and order.

2. (3P) We consider a recursive algorithm that involves an input with \(n\) objects and having running time \(T(n) = 4 \cdot T(n/2) + n^2\). Determine the running Theta class of the algorithm.

Solution: Identify recurrence: \(T(n) = a \cdot T(n/b) + f(n)\). Apply Master Theorem: Compare \(f(n)\) with \(n^{\log_b a}\) and determine the case.

3. (3P) We consider a recursive algorithm that involves an input with \(n\) objects and having running time \(T(n) = 4 \cdot T(n/2) + n\). Determine the running Theta class of the algorithm.

Solution: Apply Master Theorem: Use Case 1 for \(f(n) = O(n^{\log_b a - \epsilon})\).

4. (3P) We consider a recursive algorithm that involves an input with \(n\) objects and having running time \(T(n) = 4 \cdot T(n/2) + n^3\). Determine the running Theta class of the algorithm.

Solution: Apply Master Theorem: Use Case 3 for \(f(n) = \Omega(n^{\log_b a + \epsilon})\).

\subsubsection*{Solution:}
1. \(\Theta(\sin n), \Theta(\log_2 n), \Theta(\ln^2 n), \Theta(n^2), \Theta(e^{\log_2 n}), \Theta(n^3 + \ln(n^2)), \Theta(e^{\ln(n^4)}), \Theta(e^{n^2})\)

2. Case 2 Master theorem. \(\Theta(n^2 \ln n)\).

3. Case 1 Master theorem. \(\Theta(n^2)\).

4. Case 3 Master theorem. \(\Theta(n^3)\). (In all three cases, 1p corrected answer 2p correct justification)

\subsubsection*{Example 2}
1. (4P) Write in increasing order from left to right the following list of Theta classes. When two classes are equal, make it clear.

\(\Theta(n^{4/5}), \Theta(100n), \Theta(n \log_4 n), \Theta(n^{3/2}), \Theta(n^{\sqrt{n}}), \Theta(e^{\sqrt{2} \cdot n}), \Theta(e^n), \Theta(n!)\)

Solution: Understand growth rates: Familiarize yourself with common functions like polynomials, logarithms, exponentials, and factorials. Use limits to compare and order.

2. (2P) Give two functions \(f(n)\) and \(g(n)\) such that

\(\Theta(f(n) + g(n)) \neq \Theta(f(n))\) and \(\Theta(f(n) + g(n)) \neq \Theta(g(n))\)

Solution: Choose functions with different growth rates.

\subsubsection*{Solution:}
\(\Theta(n^{4/5}), \Theta(100n), \Theta(n \log_4 n), \Theta(n^{\sqrt{n}}), \Theta(n^{3/2}) = \Theta(n^{\sqrt{n}}), \Theta(e^n), \Theta(e^{\sqrt{2} \cdot n}), \Theta(n!)\)

\(f(n) = n\) and \(g(n) = -n + 1\)

\subsection*{Exercise 2: Pseudocode Examples}

\subsubsection*{Example 1: Pseudocode}

1. (6P) Write a pseudocode, which takes the input $A = \{a_1,a_2,\ldots,a_n\}$ and gives you as output $B = \{a_n,\ldots,a_2,a_1\}$ (i.e. same entries as in $A$ but, in the reversed order).

Solution approach:
- Use array indexing to swap elements
- Only need to iterate through half the array (why? because each swap handles two positions)
- Track running time: each operation is constant time, done n/2 times

\begin{verbatim}
REVERSEARRAY(A)
for i = 1 to ⌊n/2⌋
    exchange A[i] with A[n-i+1]
return A
\end{verbatim}

Running time: $\Theta(n)$ because we perform n/2 constant-time operations.

2. (9P) Let $S$ be a set with $n$ integers, randomly ordered and not necessarily distinct. Let $m$ be an integer. Write a pseudocode, which tests whether there are two elements $a,b \in S$ with $a + b = m$ (a and b may be the same number).

Solution approach:
- Naive solution: Check all pairs with nested loops = $O(n^2)$
- Better solution: Sort first, then use two pointers = $O(n \log n)$
- Best solution: Single pass with smart pointer movement = $O(n)$

Naive solution (for understanding):
\begin{verbatim}
SUMTEST(S,m)
for i = 1 to n-1
    for j = i+1 to n
        if S[i] + S[j] = m return true
return false
\end{verbatim}

Optimized solution:
\begin{verbatim}
SUMTESTBEST(S,m)
i = 1 to j = n
if S[i] + S[j] = m return true
if S[i] + S[j] < m, i=i+1
else j = j+1 (# that is if S[i] + S[j] > m)
stop when returns true or i > j
\end{verbatim}

Key insights for optimization:
- Avoid checking all pairs
- Use array properties to skip unnecessary checks
- Think about how to move pointers based on sum comparison with target
- Running time improves from $O(n^2)$ to $O(n)$

\subsection*{Exercise 3: Sweep Line Algorithm}

\subsubsection*{Example 1: ANYSEGMENTINTERSECT}

Given the algorithm ANYSEGMENTINTERSECT and line segments on an (x,y)-plane, determine the sweep lines and partial orders.

Solution approach:
1. Understand the algorithm:
   - Sorts endpoints from left to right
   - Maintains partial order of segments crossing sweep line
   - Checks for intersections at each event point

2. Steps to solve:
   a) First, identify all endpoints (left and right) of segments
   b) Sort them from left to right
   c) For each vertical sweep line at these points:
      - Draw the vertical line
      - List segments crossing this line from bottom to top
      - Check for intersections between adjacent segments

3. Key points to remember:
   - Left endpoint: INSERT segment into order
   - Right endpoint: DELETE segment from order
   - Check ABOVE and BELOW neighbors for intersections
   - Stop when intersection found

4. Example solution format:
   Sweep line at x = -2:
   Segments (bottom to top): a, f
   
   Sweep line at x = -1:
   Segments: a, e, f, b
   
   Sweep line at x = 0:
   Segments: c, e, f, b

Running time analysis:
- Sorting endpoints: $O(n \log n)$
- Each endpoint processed once: $O(n)$
- Total: $O(n \log n)$

Common mistakes to avoid:
- Don't forget to order segments from bottom to top
- Check both ABOVE and BELOW at insertion
- List ALL segments crossing each sweep line

\subsection*{Exercise 4: Points on the Plane}

\subsubsection*{Example 1: Points on the Plane}

1. (9P) Check for collinear points in 2D plane:

Solution steps:
a) For each point $p_0$ as center:
   - Calculate angles between $\overline{p_0p_1}$ and all other segments $\overline{p_0p_j}$
   - Use DIRECTION($p_0$, $p_i$, $p_j$) = cross product $(p_1 - p_0) \times (p_j - p_0)$
   - Calculate angle: $\alpha_j = \sin^{-1}(\frac{\text{cross product}}{||p_1-p_0|| \cdot ||p_j-p_0||})$

b) For each center point:
   - Sort angles using merge sort: $O(n \log n)$
   - Use SUMTESTBEST to find angles with 0 or $\pi$ difference: $O(n)$
   - Three points collinear if difference is 0 or $\pi$

Total running time: $O(n^2 \log n)$ because:
- Repeat for each point: $O(n)$
- For each center: $O(n \log n)$ for sorting + $O(n)$ for checking
- Final complexity: $n \cdot O(n \log n) = O(n^2 \log n)$

2. (6P) BUILDKDTREE construction:

Steps to construct KD-Tree:
a) Start with depth 0 (vertical split)
   - Find median x-coordinate
   - Split points into left/right sets

b) At depth 1 (horizontal split):
   - Find median y-coordinate in each subset
   - Split into top/bottom sets

c) Continue alternating between:
   - Even depth: vertical splits (x-coordinate)
   - Odd depth: horizontal splits (y-coordinate)

Example tree structure:
\begin{verbatim}
At root (depth 0): vertical split
- Left child: points left of median
- Right child: points right of median
  At depth 1: horizontal splits
  - Top/bottom division for each subset
\end{verbatim}

Key insights:
- Alternate between x and y coordinates based on depth
- Always split through median point
- Each split divides remaining points roughly in half
- Tree will be balanced if splits are perfect medians

\subsection*{Exercise 5: Fibonacci Sequence Analysis}

Input: Positive integer $n > 2$

Algorithm analysis:
\begin{verbatim}
INPUT: a positive integer n > 2
Create A an empty array with n entries
Set A[1] = A[2] = 1
for (i = 3; i ≤ n) do
    A[i] = A[i-1] + A[i-2]
return A[n]
\end{verbatim}

Solution steps:
- Algorithm builds Fibonacci sequence: 1, 1, 2, 3, 5, 8, 13, ...
- Each number is sum of previous two
- For n = 5, output is (1, 1, 2, 3, 5)
- Running time: $\Theta(n)$ as we do n-3 iterations with constant operations

\subsection*{Exercise 6: Binary Search Implementation}

Write a recursive algorithm (divide-and-conquer) to check if k is in sorted array A.

Solution approach:
1. Decomposition:
   - Find middle element
   - Compare k with middle
   - Determine which half to search

2. Implementation:
\begin{verbatim}
BINARYSEARCH(A, k, left, right)
    if left > right return -1
    mid = ⌊(left + right)/2⌋
    if A[mid] = k return mid
    if A[mid] > k
        return BINARYSEARCH(A, k, left, mid-1)
    return BINARYSEARCH(A, k, mid+1, right)
\end{verbatim}

Running time: $\Theta(\log n)$ because:
- Each step divides problem size by 2
- Constant work at each level
- Tree height is $\log n$

\subsection*{Exercise 7: Minimum Distance Algorithm}

Algorithm analysis for finding minimum distance between points:
\begin{verbatim}
INPUT: n ≥ 2 points (x₁,y₁), (x₂,y₂) ..., (xₙ,yₙ)
a = √((x₁-x₂)² + (y₁-y₂)²)
for (i = 1; i < n) do
    for (j = i+1; j ≤ n) do
        r = √((xᵢ-xⱼ)² + (yᵢ-yⱼ)²)
        if (r < a) then
            a = r
return a
\end{verbatim}

Solution analysis:
- Computes distance between all pairs of points
- Updates minimum distance when smaller distance found
- Running time: $\Theta(n^2)$ due to nested loops
- Space complexity: $\Theta(1)$ as only storing minimum distance

\subsection*{Exercise 8: Dijkstra's Algorithm}

\subsubsection*{Example 1: Dijkstra's Algorithm}

Consider weighted oriented graph with adjacency matrix:
\[
\begin{pmatrix}
    & s & x & y & z & w \\
s & 0 & 2 & 3 & 0 & 0 \\
x & 0 & 0 & 0 & 1 & 5 \\
y & 0 & 0 & 0 & 4 & 0 \\
z & 0 & 0 & 0 & 0 & 3 \\
w & 0 & 0 & 0 & 0 & 2
\end{pmatrix}
\]

Steps for Dijkstra's Algorithm:
1. Initialize:
   - Set source distances (s: 0, others: ∞)
   - Q = V (all vertices in queue)
   - S = ∅ (empty set of processed vertices)

2. Main loop:
   - Extract min from Q
   - Add to S
   - Update neighbors' distances
   - Track predecessors

3. Key points to remember:
   - Black vertices: in S (processed)
   - White vertices: in Q (unprocessed)
   - Grey vertex: current EXTRACT-MIN(Q)
   - Shaded edges: show predecessor values

\subsection*{Exercise 9: KD-Tree Construction}

\subsubsection*{Example 1: KD-Tree Construction}

Given points in xy-plane: $(-5,5), (-4,3), (-3,-5), (-2,-1), (-1,0), (0,-3), (1,2), (2,-4), (3,6)$

Steps to construct KD-Tree:
1. Depth 0 (vertical split):
   - Sort by x-coordinate
   - Find median x value
   - Split points into left/right sets

2. For each subset at depth 1:
   - Sort by y-coordinate
   - Split into top/bottom

3. Implementation tips:
   - Even depth: vertical split (x-coord)
   - Odd depth: horizontal split (y-coord)
   - Always include median in split
   - Balance tree by choosing true median

4. Visualization:
   - Draw vertical/horizontal lines at splits
   - Show points in resulting regions
   - Connect nodes based on splits
   - Label depth and split direction

\subsection*{Example 7: Array Operations}

1. Check for duplicates in array:
\begin{verbatim}
DUPLICATES(A)
    found = false
    for i from 1 to n-1
        for j=i+1 to n
            if a[i]=a[j] found=true
    return found
\end{verbatim}
Running time: $\Theta(n^2)$ because we check $\frac{n(n-1)}{2}$ pairs

2. Recursive sum of array segment:
\begin{verbatim}
Sum(a,l,r)
    if l=r
        return a[l]
    else
        centre=⌊(l+r)/2⌋
        sum1 = Sum(a,l,c)
        sum2 = Sum(a,c+1,r)
        return sum1+sum2
\end{verbatim}
Running time analysis:
- T(1) = c (constant time)
- T(n) = 2T(n/2) + O(1)
- By Master Theorem (Case 1): T(n) = Θ(n)

\subsection*{Example 8: Binary Search Tree Operations}

Tree-Delete operation steps:
1. Find node to delete
2. If leaf: simply remove
3. If one child: replace with child
4. If two children:
   - Find successor (minimum in right subtree)
   - Replace node with successor
   - Delete successor from original position

Key insights:
- Maintain BST properties after deletion
- Handle all cases (0, 1, 2 children)
- Track parent pointers for clean deletion

\subsection*{Example 9: Segment Intersections}

For segments between two parallel lines (y=0 and y=1):

Algorithm approach:
1. Sort points by x-coordinate
2. Count inversions to find intersections
3. Use divide-and-conquer:
   - Split into halves
   - Count inversions in each half
   - Merge and count cross-inversions

Running time analysis:
- T(n) = 2T(n/2) + O(n)
- By Master Theorem: T(n) = O(n log n)

Key optimization:
- Sort once at beginning
- Use merge-sort principle for counting
- Avoid checking all pairs (would be O(n²))
