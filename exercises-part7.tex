\section{Exercises Part 2}
\subsection{Exercise 8.1: Drawing 5 Lines}
\textbf{Problem:} Prove that you cannot draw 5 lines on the Euclidean plane in such a way that each line cuts exactly 3 other lines.

\textbf{Solution:}
Using graph modeling:
\begin{itemize}
\item One line = one node
\item One intersection = one edge
\item Problem: Draw a graph with 5 nodes, each of degree 3
\item Sum of the degrees = 15
\item In any graph, sum of degrees must be even (each edge has 2 extremities)
\item Conclusion: Infeasible!
\end{itemize}

\textbf{Explanation:}
To solve similar problems, consider the following steps:
\begin{enumerate}
\item Model the problem using graph theory concepts.
\item Identify nodes and edges based on the problem statement.
\item Calculate the degree of each node and ensure the sum of degrees is even.
\item Use these calculations to determine feasibility.
\end{enumerate}

\subsection{Exercise 8.2: TSP Mathematical Formulation}
\textbf{Problem:} Give an exact formulation of the travelling salesperson problem.

\textbf{Solution:}
Input data:
\begin{itemize}
\item Distance matrix $D = (d_{ij})$ between cities $i$ and $j$
\item Objective: Find permutation $p$ of $n$ cities minimizing:
\[ \sum_{i=1}^{n-1} d_{p_ip_{i+1}} + d_{p_np_1} \]
\end{itemize}

\textbf{Explanation:}
For similar exercises:
\begin{enumerate}
\item Understand the problem constraints and objectives.
\item Define the input data clearly, such as distance matrices.
\item Formulate the objective function based on the problem requirements.
\item Use permutations or combinations to find optimal solutions.
\end{enumerate}

\subsection{Exercise 8.3: Cards Problem Optimization}
\textbf{Problem:} 50 cards numbered 1 to 50 must be separated into 2 stacks. First stack sum = 1170, second stack product = 36000.

\textbf{Solution:}
Encoding:
\begin{itemize}
\item Zero-one vector $s$ where $s_i = 0$ means card $i$ is in first stack
\item Objective function to minimize:
\[ f(s) = \left|\frac{1170 - \sum_{i=1}^{50} i \cdot (1-s_i)}{1170}\right| + \left|\frac{36000 - \prod_{i=1}^{50} i^{s_i}}{36000}\right| \]
\end{itemize}

\textbf{Explanation:}
For similar optimization problems:
\begin{enumerate}
\item Define the decision variables and constraints clearly.
\item Formulate the objective function to reflect the problem goals.
\item Use mathematical tools to solve for optimal solutions.
\item Validate results by checking against constraints.
\end{enumerate}

\subsection{Exercise 8.4: Timetable for Exams}
\textbf{Problem:} Timetable for Exams

\textbf{Solution:}
This problem can be formulated as a Vertex Colouring Problem: Each examination is a node, and two nodes are connected if at least one student exists who has to take both exams.

\textbf{Steps to Solve:}
\begin{enumerate}
\item \textbf{Identify Nodes:} Each exam is a node in the graph.
\item \textbf{Connect Nodes:} Draw an edge between nodes if a student is enrolled in both exams.
\item \textbf{Apply Colouring:} Use colouring to assign days to exams such that no two connected nodes share the same colour.
\item \textbf{Minimize Colours:} The goal is to use the minimum number of colours, representing the minimum number of days.
\end{enumerate}

\subsection{Exercise 8.5: Permutation Flow Shop Problem}
\textbf{Problem:} Number of Solutions for Permutation Flow Shop Problem

\textbf{Solution:}
For Permutation Flow Shop Problems, only the ordering of the n jobs on the first machine need to be selected, since the ordering of tasks on the other machines is then fixed. So there are n! permutations of n jobs.

\textbf{Steps to Solve:}
\begin{enumerate}
\item \textbf{Understand the Problem:} Recognize that the order of jobs on the first machine determines the sequence for all machines.
\item \textbf{Calculate Permutations:} Use factorial $n!$ to calculate the number of possible job sequences.
\item \textbf{Apply Constraints:} Consider any additional constraints that might affect the ordering.
\end{enumerate}

\subsection{Exercise 8.6: Asymptotic Runtime}
\textbf{Problem:} Asymptotic Runtime

\textbf{Solution:}
a) The asymptotic runtime of the given code is $O(n^2)$
b) The ascending ordering is: 1, log n, n, $n^2$, $n^3$, $(3/2)^n$, $2^n$.

\textbf{Steps to Solve:}
\begin{enumerate}
\item \textbf{Analyze Code:} Break down the loops to understand their contribution to runtime.
\item \textbf{Identify Dominant Terms:} Focus on the terms that grow fastest as $n$ increases.
\item \textbf{Order Functions:} Arrange functions by growth rate to understand their asymptotic behavior.
\end{enumerate}

\subsection{Exercise 8.7: Dijkstra's Algorithm}
\textbf{Problem:} Example on Dijkstra's Algorithm

\textbf{Solution:}
A, B, C, D, E
lambda = (0, 5, 2, 10, 12)
p = (-, C, A, B, D)

\textbf{Steps to Solve:}
\begin{enumerate}
\item \textbf{Initialize Distances:} Set the starting node's distance to zero and all others to infinity.
\item \textbf{Select Node:} Choose the node with the smallest tentative distance.
\item \textbf{Update Neighbors:} Calculate the tentative distances for neighboring nodes.
\item \textbf{Repeat:} Continue until all nodes are processed.
\item \textbf{Construct Path:} Use the predecessor array to reconstruct the shortest path.
\end{enumerate}

\subsection{Exercise 8.8: Prim's Algorithm}
\textbf{Problem:} Example on Prim's Algorithm

\textbf{Solution:}
A, B, C, D, E, F, G
lambda = (0, 7, 5, 5, 7, 6, 9)
p = (-, A, E, A, B, D, E)
$E_T$ = \{(A,D), (D,F), (A,B), (B,E), (E,C), (E,G)\}

\textbf{Steps to Solve:}
\begin{enumerate}
\item \textbf{Initialize:} Start with a single node and an empty edge set.
\item \textbf{Select Edge:} Choose the smallest edge connecting the tree to a new vertex.
\item \textbf{Add Edge:} Include this edge and vertex in the tree.
\item \textbf{Repeat:} Continue until all vertices are included in the tree.
\end{enumerate}
